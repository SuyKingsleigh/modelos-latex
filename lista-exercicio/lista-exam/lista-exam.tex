\documentclass[12pt,addpoints]{exam}
\usepackage{estilo-lista-exam}

% -------------------------------------------------------------------------------------------------------------- %
% Descomente a linha abaixo se quiser usar um formato diferente para apresentação do enunciado da questão
%\formatoquestao
% -------------------------------------------------------------------------------------------------------------- %



% -------------------------------------------------------------------------------------------------------------- %
% Descomente o comando \printanswers a seguir para gerar o gabarito com soluções
% -------------------------------------------------------------------------------------------------------------- %
% \printanswers

 % As respostas podem aparecer dentro de uma caixa ou com o fundo preenchido. Escolha uma das opções abaixo
\shadedsolutions
\definecolor{SolutionColor}{rgb}{0.85,0.9,1}
%\framedsolutions
% -------------------------------------------------------------------------------------------------------------- %


\begin{document}

%\listacomnomealuno{1}{Assunto}{20/06/2017}{Nome da Disciplina}{Emerson Ribeiro de Mello}
\listasemnomealuno{1}{Assunto}{20/06/2017}{Nome da Disciplina}{Emerson Ribeiro de Mello}


\begin{questions}

\question
Qual a cor do céu?

\begin{solutionorlines}[2cm]
Geralmente é azul.
\end{solutionorlines}


\question
Calcule $\displaystyle\int_0^1 x \, dx$.

\begin{solutionorbox}[2cm]
$\displaystyle\tfrac{x^2}{2}\Big|^1_0 = \tfrac{1^2}{2} - \tfrac{0^2}{2} = \tfrac{1}{2}$
\end{solutionorbox}

\question
Quantos meses possuem no máximo 30 dias?
\begin{solutionorlines}[2cm]
Somente 4. São eles: abril, junho, setembro e novembro.
\end{solutionorlines}


\question
  Sobre o sistema operacional Linux
  \begin{parts}
    \part
    Liste três aplicativos utilitários que vem por padrão
	\begin{solutionorlines}[1cm]
	ls, cd e mv.
	\end{solutionorlines}
    \part
	Liste o nome de três distribuições
	\begin{solutionorlines}[1cm]
	Debian, Redhat e Ubuntu. 
	\end{solutionorlines}
  \end{parts}


\question
Quais dos meses abaixo pode ter um número de dias diferente em anos bissextos?
\begin{choices}
	\choice Janeiro
	\correctchoice Fevereiro
	\choice Abril
	\choice Junho
\end{choices}

\end{questions}


\end{document}
